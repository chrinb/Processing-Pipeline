\newpage 
\subsection{Session}
\begin{table}[ht]
\noindent\begin{tabular*}{\linewidth}{@{\extracolsep{\fill} } llll}
\multicolumn{4}{l}{\bfseries Session Info} \\
\hline
Date: & 08.01.2017 & Session Protocol: & Lights 90 deg/s rotations \\
Nblocks: & 20 & Duration: & 20 min \\
Ntrials per block: & 10 & Intertrial interval: & 5 sec \\
\\
\end{tabular*}
\noindent\begin{tabular*}{\linewidth}{@{\extracolsep{\fill} } llllll}
\multicolumn{6}{l}{\bfseries Microscope Info} \\
\hline
PMT ch1: & 567 & PMT ch2: & 660 & Laser Power: & 60mW \\
\\
\multicolumn{6}{l}{\bfseries Population Info} \\
\hline
\end{tabular*}
\begin{minipage}[t]{0.55\linewidth}
\vspace{1mm}
\includegraphics[width=\textwidth]{/Users/eivinhen/Google" "Drive/PhD/Lab/Eivind" "Hennestad/RotationExperiments/LabBook/LabBook_tex/populations/roi_reference.png}
\end{minipage}
\begin{minipage}[t]{0.45\linewidth}
\hfill \begin{tabular}[t]{ll}
\\
Population: & 1 \\
Location: & 2 mm, 0.2 mm \\
Depth: & 500 um \\
Number of cells: & 20 \\
FOV Size: & 400 x 400 um \\
Frame movement: & Compound variable?\\
\end{tabular}
\end{minipage}
\end{table}

\noindent\begin{tabular*}{\linewidth}{@{\extracolsep{\fill} } l}
{\bfseries Activity Summary} \\
\hline
\end{tabular*}
\includegraphics[width=\linewidth]{psth.png}
Show figure of identified event for all cells and all blocks, like a PSTH?
